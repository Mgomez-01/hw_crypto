% Created 2024-01-31 Wed 10:58
% Intended LaTeX compiler: pdflatex
\documentclass[11pt]{article}
\usepackage[utf8]{inputenc}
\usepackage[T1]{fontenc}
\usepackage{graphicx}
\usepackage{longtable}
\usepackage{wrapfig}
\usepackage{rotating}
\usepackage[normalem]{ulem}
\usepackage{amsmath}
\usepackage{amssymb}
\usepackage{capt-of}
\usepackage{hyperref}
\usepackage[T1]{fontenc}
\usepackage{titling}
\usepackage{url}
\usepackage{amsmath,amsthm,amssymb}
\documentclass[a4paper, 11pt]{exam}
\usepackage[T1]{fontenc}
\usepackage{titling}
\usepackage{url}
\usepackage{amsmath,amsthm,amssymb}
\usepackage{graphicx}
\usepackage{graphics}
\usepackage{listings}
\usepackage[dvipsnames]{xcolor}
\usepackage{tabularx}
\usepackage{ragged2e}
\usepackage{courier}
\usepackage{textcomp}
\usepackage{circuitikz}
\usepackage{tikz}
\usepackage{enumitem}
\usepackage{karnaugh-map}
\usepackage{bytefield}
\usepackage{mathrsfs}
\usepackage{cancel}
\usepackage[linesnumbered,ruled,vlined]{algorithm2e}
\usepackage{hyperref}
\usepackage{environ}
\usepackage{listings}
\usepackage{algorithm}
\usepackage{algpseudocode}
\lstset{breaklines=true, basicstyle=\ttfamily\tiny, frame=single}
\author{Miguel Gomez U1318856}
\date{2024-01-31 10:58:02}
\title{Homework Assignment \# 1}
\hypersetup{
 pdfauthor={Miguel Gomez U1318856},
 pdftitle={Homework Assignment \# 1},
 pdfkeywords={},
 pdfsubject={},
 pdfcreator={Emacs 28.2 (Org mode 9.4.6)}, 
 pdflang={English}}
\begin{document}

\maketitle
\tableofcontents



\section{Homework 1}
\label{sec:org3a8c9a5}
The extended euclidean algorithm is a spin on the usual algo that allows for us to split the number into two intermediate values. one that is equal to the number we are dividing in the algo multiplied by a number, and another that is the divisor multiplied by the quotient. This expression below is the one we end up with:
\subsection{expression}
\label{sec:org4585130}
\[
\begin{align*}
g &= gcd(a,b)\\
\exists\ s,t\ |\ s\cdot a + t\cdot b &= g
\end{align*}
\]



\begin{verbatim}
start=$(date +%s.%N)
Singular sing/hw1_b.sing | grep -v -e "\*\* loaded\|\*\* library"
end=$(date +%s.%N)
echo "Execution Time: $(echo "$end - $start" | bc) seconds"
\end{verbatim}


\[
\subsection{output of hw$1_b$ results}
\begin{lstlisting}[language=Singular]
                     SINGULAR                                 /  Development
 A Computer Algebra System for Polynomial Computations       /   version 4.2.1
                                                           0<
 by: W. Decker, G.-M. Greuel, G. Pfister, H. Schoenemann     \   May 2021
FB Mathematik der Universitaet, D-67653 Kaiserslautern        \  Debian 1:4.2.1-p3+ds-1
// ** but for functionality you may wish to change to the new
// ** format. Please refer to the manual for further information.
The example computed GCD of 24 and 16 is:
8
// ** redefining r (ring r = integer, (x), lp;) hw1_b.sing:21
The computed myintGCD of 24 is: 8
The computed myEuclid of 24 is: 8
The computed myExtendedEuclid of the numbers is:

GCD(24,16) = 8
s = 1
t = -1
    
The computed GCD of the list of numbers for problem 1-b is:
10
Auf Wiedersehen.
Execution Time: .028753389 seconds
\end{lstlisting}
\]


\[
\newpage
\subsection{Pseudocode for the Euclidean algo}
\begin{algorithm}
\caption{Euclidean Algorithm}
\begin{algorithmic}[1]
\Procedure{myExtendedEuclid}{$a$, $b$}
    \State $R1 \gets a$
    \State $R2 \gets b$
    \While{$R2 \neq 0$}
        \State $Q \gets (R1/R2)$
        \State $r \gets R1 - Q \times R2$
        \State $R1 \gets R2$
        \State $R2 \gets r$
    \EndWhile
    \State \textbf{return} $r$
\EndProcedure
\end{algorithmic}
\end{algorithm}
\]


\[
\newpage
\subsection{Pseudocode for the Euclidean algo}
\begin{algorithm}
\caption{Extended Euclidean Algorithm}
\begin{algorithmic}[1]
\Procedure{myExtendedEuclid}{$a$, $b$}
    \State $R1 \gets a$
    \State $R2 \gets b$
    \State $S1 \gets 1$
    \State $S2 \gets 0$
    \State $T1 \gets 0$
    \State $T2 \gets 1$
    \While{$R2 > 0$}
        \State $Q \gets \text{floor}(R1/R2)$
        \State $r \gets R1 - Q \times R2$
        \State $R1 \gets R2$
        \State $R2 \gets r$
        \State $s \gets S1 - Q \times S2$
        \State $S1 \gets S2$
        \State $S2 \gets s$
        \State $t \gets T1 - Q \times T2$
        \State $T1 \gets T2$
        \State $T2 \gets t$
    \EndWhile
    \State \textbf{print} "GCD(", $a$, ",", $b$, ") = ", $S1 \times a + T1 \times b$
    \State \textbf{print} "s = ", $S1$
    \State \textbf{print} "t = ", $T1$
    \State $L \gets \text{list}()$
    \State $L \gets \text{list}(S1 \times a + T1 \times b, S1, T1)$
    \State \textbf{return} $L$
\EndProcedure
\end{algorithmic}
\end{algorithm}
\newpage
\]


\subsection{Identify whether the integers 38 and 7 have multiplicative inverses in \mathcal{Z}\textsubscript{180}}
\label{sec:orgd73733f}
Since the number \(p\) we are working with is even, we will not have multiplicative inverses for even numbers. Therefore we only need to find the inverse for the one we can, for 7. 

\[
a \in \mathcal{Z}_{180}\ ,\ a^{-1} \in \mathcal{Z}_{180}\ \text{if}\ gcd(a,180) = 1
\]

\begin{verbatim}
start=$(date +%s.%N)
Singular sing/hw1_c.sing | grep -v -e "\*\* loaded\|\*\* library"
end=$(date +%s.%N)
echo "Execution Time: $(echo "$end - $start" | bc) seconds"
\end{verbatim}

\[
\subsection{output of hw$1_c$ results}
\begin{lstlisting}[language=Singular]
                     SINGULAR                                 /  Development
 A Computer Algebra System for Polynomial Computations       /   version 4.2.1
                                                           0<
 by: W. Decker, G.-M. Greuel, G. Pfister, H. Schoenemann     \   May 2021
FB Mathematik der Universitaet, D-67653 Kaiserslautern        \  Debian 1:4.2.1-p3+ds-1
// ** but for functionality you may wish to change to the new
// ** format. Please refer to the manual for further information.
The computed myintGCD of 7 is:
1
The computed myintGCD of 38 is:
2

GCD(7,180) = 1
s = -77
t = 3
    
The inverse of 7 modulo 180 is 103

GCD(38,180) = 2
s = 19
t = -4
    
38 has no inverse modulo 180
Auf Wiedersehen.
Execution Time: .022770642 seconds
\end{lstlisting}
\]


Since the gcd for the expression comes out to 1, the inverse exists and is printed out. Since 38 is even, no inverse is possible modulo 180.

\section{Problem 2}
\label{sec:orgba1025a}
Solving linear diophantine equations using linear congruences.
\subsection{a) solving LC \(4x \equiv 4\ \text{mod}\ 6\)}
\label{sec:org71924a8}
Solving this is easiest with a table of the results we would get by plugging in any values for x from the set mod 6.

\begin{center}
\begin{tabular}{|c|c|c|}
\hline
\( x \) & \( 4x \mod 6 \) & Congruent to 4? \\
\hline
0 & \( 4 \cdot 0 \mod 6 = 0 \) & No \\
1 & \( 4 \cdot 1 \mod 6 = 4 \) & Yes \\
2 & \( 4 \cdot 2 \mod 6 = 2 \) & No \\
3 & \( 4 \cdot 3 \mod 6 = 0 \) & No \\
4 & \( 4 \cdot 4 \mod 6 = 4 \) & Yes \\
5 & \( 4 \cdot 5 \mod 6 = 2 \) & No \\
\hline
\end{tabular}
\end{center}
\(\therefore\) we have exactly two solutions which are congruent for this problem.

\subsection{Solving as an LDE instead}
\label{sec:org1d80056}
using the expression \(4x \equiv 4\ \text{mod}\ 6\), we can transform the expression into the following:
\begin{align*}
4x &\equiv 4\ \text{mod}\ 6\\
6 &|\ 4x - 4\\
6k &= 4(x - 1)\\
3k &= 2(x - 1)
\end{align*}
Now we find values of $x$ that would allow the expression to be integer valued when $x \in \{0..5\}$. In general, the solutions will be the same as they were before giving us just two possible solutions to the expression. Using the following:
\begin{align*}
 x &= 1\\
 3k &= 2(1 - 1) = 0\\
 k &= 0 \\
 x &= 4\\
 3k &= 2(4 - 1) = 6\\
 k &= 2 
\end{align*}

\section{Problem 3}
\label{sec:org7d156fd}
We must find a solution to the affine cipher and obtain the keys \([k_1, k_2]\). We can set this up in singular to solve for the key vector using the inverse matrix algorithm that utilizes the transformation to a reduced row eschelon form of the matrix.



\begin{verbatim}
start=$(date +%s.%N)
Singular sing/hw3.sing | grep -v -e \
	    "\*\* loaded\|\*\* library\|\*\* redefining"
end=$(date +%s.%N)
echo "Execution Time: $(echo "$end - $start" | bc) seconds"
\end{verbatim}


\[
\subsection{output of hw$3$ results}
\begin{lstlisting}[language=Singular]
                     SINGULAR                                 /  Development
 A Computer Algebra System for Polynomial Computations       /   version 4.2.1
                                                           0<
 by: W. Decker, G.-M. Greuel, G. Pfister, H. Schoenemann     \   May 2021
FB Mathematik der Universitaet, D-67653 Kaiserslautern        \  Debian 1:4.2.1-p3+ds-1
print matrix A
18,1,
19,1 
print matrix B
4,
19
Determinant of A:
printing det(A)
25
gcd(det(A), 26) is:
1
inverse of A exists
inverse of A:
inv_A[1,1]=25
inv_A[1,2]=1
inv_A[2,1]=19
inv_A[2,2]=8
check of inv_A*A = I:
_[1,1]=1
_[1,2]=0
_[2,1]=0
_[2,2]=1
Solutions for x = :
_[1,1]=15
_[2,1]=20
Auf Wiedersehen.
Execution Time: .053407240 seconds
\end{lstlisting}
\]
\end{document}